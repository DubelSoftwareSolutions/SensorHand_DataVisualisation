\documentclass[12pt,a4paper]{article}
\usepackage[utf8x]{inputenc}
\usepackage{ucs}
\usepackage[MeX]{polski}
\usepackage{fancyhdr}
\usepackage{amsmath}
\usepackage{amsfonts}
\usepackage{amssymb}
\pagestyle{fancy}
\usepackage{enumerate}
\usepackage{listings}
\usepackage{subfig}

\begin{document}
\LARGE\centering Założenia projektowe\\
\large\centering Projekt realizowany w ramach kursu Wizualizacja Danych Sensorycznych na Politechnice Wrocławskiej\\
\vspace{5 mm}
\normalsize\flushleft\textbf{Temat Projektu:} Wizualizacja czujników rękawicy sensorycznej\\
\textbf{Autorzy:} Krzysztof Dąbek 218549, Dymitr Choroszczak 218627\\
\textbf{Kierunek:} Automatyka i Robotyka\\
\textbf{Specjalność:} Robotyka (ARR)\\
\textbf{Prowadzący:} dr inż. Bogdan Kreczmer\\
\textbf{Kurs:} Wizualizacja Danych Sensorycznych\\
\textbf{Termin zajęć:} pt 11:15\\
\vspace{5 mm}

\section{Główne założenia projektowe:}
\begin{itemize}
\item Stworzenie aplikacji okienkowej do wizualizacji z użyciem biblioteki Qt
\item Stworzenie uproszczonego modelu kośćca dłoni
\item Wizualizacja Danych z tensometrycznych czujników ugięcia za pomocą stworzonego modelu na podstawie wyliczonej kinematyki manipulatora z przegubami rotacyjnymi
\item Wizualizacja Danych z czujników nacisku za pomocą kolorowych sfer na opuszkach palców
\item Wizualizacja Danych z akcelerometru za pomocą rotacji stworzonego modelu
\end{itemize}
Projekt zostanie połączony z innym realizowanym w ramach kursu Roboty Mobilne 1. Dane do wizualizacji będą wysyłane przez płytkę wykonanej rękawicy sensorycznej.

\section{Harmonogram}

\end{document}