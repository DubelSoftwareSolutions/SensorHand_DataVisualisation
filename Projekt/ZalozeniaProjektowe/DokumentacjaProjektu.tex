\documentclass[12pt,a4paper]{article}
\usepackage[utf8x]{inputenc}
\usepackage{ucs}
\usepackage[MeX]{polski}
\usepackage{fancyhdr}
\usepackage{amsmath}
\usepackage{amsfonts}
\usepackage{amssymb}
\pagestyle{fancy}
\usepackage{enumerate}
\usepackage{listings}
\usepackage{subfig}

\begin{document}
\LARGE\centering Założenia projektowe\\
\large\centering Projekt realizowany w ramach kursu Wizualizacja Danych Sensorycznych na Politechnice Wrocławskiej\\
\vspace{5 mm}
\normalsize\flushleft\textbf{Tytuł Projektu:} Wizualizacja czujników rękawicy sensorycznej\\
\textbf{Autorzy:} Krzysztof Dąbek 218549, Dymitr Choroszczak 218627\\
\textbf{Kierunek:} Automatyka i Robotyka\\
\textbf{Specjalność:} Robotyka (ARR)\\
\textbf{Prowadzący:} dr inż. Bogdan Kreczmer\\
\textbf{Kurs:} Wizualizacja Danych Sensorycznych\\
\textbf{Termin zajęć:} pt 11:15\\
\vspace{5 mm}

\section{Opis projektu:} \normalsize
Celem jest wizualizacja uproszczonego modelu dłoni na podstawie danych z rękawicy sensorycznej. 
Efektem końcowym jest przedstawienie ułożenia dłoni w przestrzeni trójwymiarowej. \\
\vspace{1cm}
Projekt skupia się na ukazaniu:
\begin{itemize}
\item Zgięcia trzech palców przez zmianę konfiguracji przegubów modelu
\item Siły nacisku opuszków na powierzchnię poprzez zmianę koloru i/lub rozmiaru obiektów sferycznych, umieszczonych na zakończeniach skrajnych przegubów modelu
\item Orientacji dłoni względem wektora grawitacji
\end{itemize} 
Projekt zostanie połączony z innym realizowanym w ramach kursu Roboty Mobilne 1. Dane do wizualizacji będą wysyłane przez płytkę wykonanej rękawicy sensorycznej.


\section{Główne założenia projektowe:}
\begin{itemize}
\item Stworzenie aplikacji okienkowej do wizualizacji z użyciem biblioteki Qt
\item Stworzenie uproszczonego modelu kośćca dłoni
\item Użyte czujniki rękawicy sensorycznej:
\begin{itemize}
\item Tensometryczne czujniki ugięcia, do wyznaczenia konfiguracji przegubów dłoni
\item Czujniki nacisku umieszczone na opuszkach palców, do pomiaru siły nacisku na powierzchnię
\item Akcelerometr, do wyznaczenia orientacji dłoni sensorycznej względem wektora grawitacji
\end{itemize}
\item Odbieranie danych z STM32DiscoveryF3 przez USB i/lub Bluetooth
\end{itemize}

%Rozpisać na kolejne tygodnie
%\section{Harmonogram}
%%\begin{itemize}
%\item Uruchomienie i przetestowanie pętli USB$\rightarrow $UART$\rightarrow $USB w celu symulacji danych sensorycznych
%\item Wczytywanie i dekodowanie danych z rękawicy sensorycznej
%\item Stworzenie struktur danych wykorzystywanych w aplikacji (przeguby, manipulatory, scena)
%\item Stworzenie okna programu i wszystkich jego funkcjonalności
%\item Stworzenie uproszczonego modelu kośćca dłoni
%\item Stworzenie elementów potrzebnych do wizualizacji czujników nacisku (kolorowe sfery na opuszkach)
%\item Poruszanie przegubami na podstawie odczytów z tensorów i kinematyki prostej manipulatorów
%\item Zmiana koloru i/lub wielkości sfer na podstawie odczytów z czujników nacisku
%\item Obrót modelu na podstawie akcelerometru
%\item Testy aplikacji
%\item Naprawianie błędów
%\item Wizualne ulepszenie aplikacji
%\end{itemize}
\end{document}